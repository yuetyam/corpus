臣某言:伏以佛者,夷狄之一法耳。自後漢時流入中國,上古未嘗有也。昔者,黃帝在位百年,年百一十歲;少昊在位八十年,年百歲;顓頊在位七十九年,年九十八歲;帝嚳在位七十年,年百五歲;帝堯在位九十八年,年百一十八歲;帝舜及禹,年皆百歲。此時天下太平,百姓安樂壽考,然而中國未有佛也。其後,殷湯亦年百歲,湯孫太戊在位七十五年,武丁在位五十九年,書史不言其年壽所極,推其年數,蓋亦俱不減百歲。周文王年九十七歲,武王年九十三歲,穆王在位百年。此時佛法亦未入中國,非因事佛而致然也。


漢明帝時,始有佛法,明帝在位,才十八年耳。其後亂亡相繼,運祚不長。宋、齊、梁、陳、元魏已下,事佛漸謹,年代尤促。惟梁武帝在位四十八年,前後三度捨身施佛,宗廟之祭,不用牲牢,晝日一食,止於菜果。其後竟爲侯景所逼,餓死臺城,國亦尋滅。事佛求福,乃更得禍。由此觀之,佛不足事,亦可知矣!


高祖始受隋禪,則議除之。當時群臣材識不遠,不能深知先王之道、古今之宜,推闡聖明,以救斯弊。其事遂止,臣常恨焉。伏惟睿聖文武皇帝陛下,神聖英武,數千百年已來,未有倫比。即位之初,即不許度人爲僧尼、道士,又不許創立寺觀。臣常以爲高祖之志,必行於陛下之手。今縱未能即行,豈可恣之轉令盛也?


今聞陛下令群僧迎佛骨於鳳翔,御樓以觀,舁入大內。又令諸寺遞迎供養。臣雖至愚,必知陛下不惑於佛,作此崇奉,以祈福祥也。直以年豐人樂,徇人之心,爲京都士庶設詭異之觀,戲玩之具耳。安有聖明若此,而肯信此等事哉!然百姓愚冥,易惑難曉,苟見陛下如此,將謂真心事佛。皆云:「天子大聖,猶一心敬信,百姓何人,豈合更惜身命!」焚頂燒指,百十爲群,解衣散錢,自朝至暮,轉相仿效,惟恐後時,老少奔波,棄其業次。若不即加禁遏,更歷諸寺,必有斷臂臠身,以爲供養者。傷風敗俗,傳笑四方,非細事也。


夫佛本夷狄之人,與中國言語不通,衣服殊制,口不言先王之法言,身不服先王之法服,不知君臣之義,父子之情。假如其身至今尚在,奉其國命,來朝京師,陛下容而接之,不過宣政一見,禮賓一設,賜衣一襲,衛而出之於境,不令惑眾也。況其身死已久,枯朽之骨,凶穢之餘,豈宜令入宮禁?


孔子曰:「敬鬼神而遠之。」古之諸侯,行弔於其國,尚令巫祝先以桃茢祓除不祥,然後進弔。今無故取朽穢之物,親臨觀之,巫祝不先,桃茢不用,群臣不言其非,御史不舉其失,臣實恥之。乞以此骨付之有司,投諸水火,永絕根本,斷天下之疑,絕後代之惑。使天下之人,知大聖人之所作爲,出於尋常萬萬也,豈不盛哉!豈不快哉!佛如有靈,能作禍祟,凡有殃咎,宜加臣身,上天鑒臨,臣不怨悔。無任感激懇悃之至,謹奉表以聞。臣某誠惶誠恐。
